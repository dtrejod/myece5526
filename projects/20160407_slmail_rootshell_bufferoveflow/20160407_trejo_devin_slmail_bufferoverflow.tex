\documentclass[12pt]{article}
% General packages
\usepackage{amsmath, graphicx, float, tabularx, booktabs, color}
% For adjusting margin size
\usepackage[margin=1in]{geometry}
% For setting bookmarks on pdf export
\usepackage[bookmarks,bookmarksopen,bookmarksdepth=2]{hyperref}
% For codeblocks
\usepackage{listings}
\usepackage{cite}

%Define Colors
\definecolor{dkgreen}{rgb}{0,0.6,0}
\definecolor{gray}{rgb}{0.5,0.5,0.5}
\definecolor{mauve}{rgb}{0.58,0,0.82}

% Color Code
\lstset{frame=tb,
  language=ruby,
  aboveskip=3mm,
  belowskip=3mm,
  showstringspaces=false,
  columns=flexible,
  basicstyle={\small\ttfamily},
  numbers=none,
  numberstyle=\tiny\color{gray},
  keywordstyle=\color{blue},
  commentstyle=\color{dkgreen},
  stringstyle=\color{mauve},
  breaklines=true,
  breakatwhitespace=true,
  tabsize=4
}

% Image path location
\graphicspath{ {images/} }

\begin{document}

% Title information
\title{Buffer Overflow SLMail-5.5.0 Service and Gain Root Shell}
\author{Devin Trejo \tabularnewline devin.trejo@temple.edu}
\date{\today}
\maketitle

\section{Summary}
\label{sect:summary}
Today we introduce the buffer overflow vulnerability by using a known case
in the SLMail5.5.0 application released in 2001. We discuss how an exploit
is constructed and then use Python to create a script that will overflow
the memory buffer. By the end of the experiment we demonstrate that we have
full control of the EIP register and write a unique string to showcase
how we can use the input from the POP3 login prompt to overwrite this 
register. 

\section{Introduction}
\label{sect:intro}
\subsection{Background SLMail5.5.0}
\label{sec:background}
SLMail is a message management tool that was advertised towards small to 
medium sized businesses published by SeatleLabs. The software was popular 
around the year 2001 for its ease of use and ``security'' of its email 
service \cite{SeattleLabs2001}. The service was also scalable for an 
unlimited number of users to use. The software boast a number of security 
features including, ``Limiting viruses by identifying specific files or types 
not permitted to enter/leave the server, rejecting emails containing unwanted 
words, avoiding external use of server as relay for spam, reduce flow of 
junk mail (anti-spam filter), and authenticate users before they send 
mail'' \cite{SeattleLabs2001}. The last ``security'' feature was instead a 
security flaw as the password authentication had a buffer overflow 
vulnerability. The service is no longer developed as is apparent if one
were to search for SLMail on SeattleLabs' website today. 

The SLMail service is an 3rd party program bought and downloaded direct
from SeattleLabs's website and typically installed on a Windows 2k server.
The default options after a succesfull installation of SLMail
can be seen in figure~\ref{fig:deafconfigslmail}. The specific version we 
concern ourselves for this project will by \textbf{SLMail5.5.0} which has a 
known buffer overflow exploit inside the user authentication prompt. When 
logging in over POP3, an application standard protocol for retrieving 
emails from a remote server, SLMail will prompt for a user-name and 
password combination associated with the desired email. If we write our 
user-name as any string combination and a password containing a shell program 
we can setup and execute the script on the remote mail server. The referenced 
shell script will be specially crafted to open a port on the remote server, 
that gives us access to a shell that contains administrative privileges. 

For reader reference, a reliable site to download the SLMail application 
with the known vulnerability is from the Exploit Database website. Link
provided below:

\url{https://www.exploit-db.com/exploits/638/}

\begin{figure}[ht]
    \centering
    \includegraphics[width=5.5in]{images/20160407_slmail_config.png}
    \caption{Default SLMAail Port Configuration}
    \label{fig:deafconfigslmail}
\end{figure}

\subsection{Attack Approach: Fuzzing Attack}
\label{sec:approachpassbuff}

The first step for this attack is to gain more information of the
SLMail 5.5.0 service. We will implement a technique known as \textbf{fuzzing}
which will allow us to discover information such as service versions, 
buffer sizes, and in general the coding implementation of the remote
service. To begin the fuzzing process, we try to find the buffer size of the 
PASS field used by SLMail's POP3 protocol. The first step is to write a 
script that loops over an array of increasing buffer sizes trying to 
determine the full length of the input buffer size. Since we already know 
there is an buffer overflow exploit for these fields we can expect at some 
point our input to overflow the allocated buffer and crash the program. 
The idea and goal for this specific fuzzing processes is to overwrite the 
\textbf{EIP register} or the address location of the next instruction
to execute on the stack.

For this assignment we will examine the structure of the SLMail-5.5.0 
program and gain insights on its construction. The POP3 interface seen on 
port 110 is not compatible with standard the standard http protocol as is 
demonstrated in figure~\ref{fig:smailpop3http}. Instead we write a Python
script that creates a socket connection to the POP3 service and interfaces
with the server using POP3 protocol commands. 

\begin{figure}[ht]
    \centering
    \includegraphics[width=5.5in]{images/20160407_http_smail.png}
    \caption{Trying to Connect to SLMail POP3 over HTTP}
    \label{fig:smailpop3http}
\end{figure}

\subsection{Application Analysis: Immunity Debugger}
\label{sec:approachimmunity}

We will later use the Immunity Debugger, a free 3rd party application wrapped
inside of Python. The Immunity Debugger is a specially crafted debugger 
built for program exploit analysis purposes as it allows monitor of the 
program heap, and gives full access to the Assembly code of the program. 

A layout of the where the stack is for a running application can be seen in 
figure~\ref{fig:thestack}. As is seen, the stack grows up to lower memory 
addresses as it runs. When functions are called they allocated memory via 
the stack which is meant for short term variables. The stack created for
a function is shown in figure~\ref{fig:fnstack}. Note how the ESP register
keep track of the top of the stack and the EBP register keeps track of 
the bottom of the stack. Any variables a function uses will be stored
into the space labeled $<MyVar>$. 

\begin{figure}[H]
    \centering
    \includegraphics[width=5.5in]{images/windowsexploitdevel.png}
    \caption{Function Memory Layout \cite{Czumak2013}}
    \label{fig:thestack}
\end{figure}

As mentioned previously, we need to monitor the EIP register and note when 
it changes before the function call has ended properly. Inside the Immunity
Debugger there is a monitor thread that allows the end user access to CPU
registers while an active process in running. The CPU monitor will note
the state of the EAX, ECX, EDX, EBX, ESP, EBP, ESI, EDI, and EIP registers.
Each register is used by the CPU to execute instructions sets accordingly. 
The details of each register are given below.

\begin{itemize}
  \item \textbf{EAX (Accumulator Register)}: Used for ADD and SUB instructions
  \item \textbf{EBX (Base Register)}: No special purpose and in general is 
  used for catch-all available storage. 
  \item \textbf{ECX (Counter Register)}: Used for loops tracking
  \item \textbf{EDX (Data Register)}: Used for division and 
  multiplication.
  \item \textbf{ESI (Source Index)}: Used to store pointer to read-only 
  location. For example, it would point to a address in memory that 
  contains a constant string.
  \item \textbf{EDI (Destination Index)}: Used to store the storage 
  pointer for functions.
  \item \textbf{EBP (Base Pointer)}: Used for keeping track of the bottom 
  of the stack.
  \item \textbf{ESP (Stack Pointer)}: Used for keeping track of the top 
  of the stack.
  \item \textbf{EIP (Instruction Pointer)}: Points to a location in memory 
  of the next instruction to be executed for the application. 
\end{itemize}

The reason why the EIP pointer is important is because it points to the location
in memory of the next instruction set that needs to be executed. By gaining
control of the EIP pointer you can have a program execute any set of
instructions placed in a specific location in memory.

\begin{figure}[ht]
    \centering
    \includegraphics[width=2.5in]{images/function_stack.png}
    \caption{Function Stack Layout \cite{Czumak2013}}
    \label{fig:fnstack}
\end{figure}

\subsection{Test Environment}
\label{sec:testenv}
For this project we use our default test environment. We have a virtual 
private network consisting of our Windows 2000 (SP4) Server, Kali Linux
penetrating machine, and a host machine running through Oracle Virtual Box. 
The virtual network has a DHCP server running on the VM host machine. The 
IP/MAC addresses for each are provided in table~\ref{table:pentestnetwork}.

\begin{table}[H]
    \centering
    \begin{tabularx}{\textwidth}{|*{3}{>{\centering}X|}}
        \toprule
        \textbf{Platform} & \textbf{MAC ADDR} & \textbf{Platform IPv4 Address} 
        \tabularnewline \midrule
        \textbf{Kali Linux:} & 08:00:27:94:5b:ba & 192.168.56.102 
        \tabularnewline
        \textbf{Windows 2k Server:} & 08:00:27:87:29:68 & 192.168.56.105
        \tabularnewline
        \textbf{VM Host Machine:} & 08:00:27:7c:86:0d & 192.168.56.100
        \tabularnewline \bottomrule
    \end{tabularx}
    \caption{IP Configuration for SLMail Pen-test Virtual Network}
    \label{table:pentestnetwork}
\end{table}

For the majority of the project we will be running our scripts from our VM
host machine. We use our Kali Linux solely to perform active information 
probing of the target machine. Our Windows 2k server instance will be 
be a fresh install of Windows with the only other 3rd party applications being
SLMail-5.5.0, Anaconda 2.4.0, MinGw32-1.0.0, and Immunity debugger 1.85.

\section{Discussion}
\label{sect:discussion}

\subsection{Fuzzing Attack}
\label{sect:fuzzatt}
To begin the intrusion we first have to setup our SLMail server. For this
test we used default parameters as seen in figure~\ref{fig:deafconfigslmail}.
Next we conducted a NMAP scan from our Kali Linux Machine using NMAP. The
scan we performed was a full version scan using the parameters seen shown 
below.

\begin{lstlisting}[language=bash]
    $ nmap -nsV 192.168.56.105
\end{lstlisting}

\begin{figure}[ht]
    \centering
    \includegraphics[width=5.5in]{images/20160407_nmap_scan.png}
    \caption{NMAP Scan of Windows Server 2k}
    \label{fig:nmapwindows}
\end{figure}

From the scan results seen in figure~\ref{fig:nmapwindows} we can see a
multitude of open ports and the services running behind the ports. What we are
interested in is port 110 which is the standard port for POP3 operations. We
know from our research that after installing SLMail an open POP3 port will
open that contains the known buffer overflow vulnerability. The NMAP scan 
revealed a number of other services running on our Windows 2k Server instance 
but for this test we will focus on port 110. 

Next we begin fuzzing the server to determine at what point the program 
will crash. A simple Python script seen in code listing~\ref{lst:passfuzz}
will loop through an array of password buffer sizes ranging from 0 to
the size of variable $MAX\_PASS\_BUFFER\_LEN$. Running the script eventually
leads us to discover a buffer of between the size of 2600 and 2800 will crash
the program. As seen in figure~\ref{fig:passfuzz}, our iterative Python
script eventually hangs as the server no longer responds.

\begin{figure}[htbp]
    \centering
    \includegraphics[width=5.5in]{images/20160422_python_overflow.png}
    \caption{Python Script Fuzzing Password Field until Program Crash}
    \label{fig:passfuzz}
\end{figure}

\subsection{Control of EIP}
\label{sect:controlEIP}

To confirm that we are indeed crashing the program by overwriting the EIP
pointer we run SLMail program within the Immunity debugger. 

Before we can use the debugger, we need to learn how the different components
of the SLMail program interact. From our task manager, we see three SLMail
related processes: SLadmin, SLsmtp, and SLmail. The SLsmtp process
listens for incoming connections implementing the Simple Mail Transfer Protocol 
whether it be on port 25, or port 110 for POP3, and delivers any input to 
the SLmail process. SLmail is the actual vulnerable process that we will
need to analyze. SLadmin simply is the conductor of the entire SLmail 
application. To monitor SLmail, we simply attach Immunity Debugger to the 
running process. 

Next we need to fine tune our script to range between 2600 and 2800 in
finner intervals to determine the exact point at which the EIP pointer is
overwritten. Similar to code in code listing~\ref{lst:passfuzz}, we instead
range range from 2600 and set $MAX\_PASS\_BUFFER\_LEN = 2800$ with a interval
of 1. 

\begin{figure}[htbp]
    \centering
    \includegraphics[width=5.5in]{images/20160501_buffer2606.png}
    \caption{Buffer = 2606 - Lower 6 Bytes of EBP Overwritten 
      in Immunity Debugger}
    \label{fig:ebpoverwrite}
\end{figure}

Starting when we pass 2605 As into the buffer we see the lower 4 bytes of 
the EBP buffer containing $4141$. By passing 2607 As we see the entire EBP
buffer containing As. Recall from figure~\ref{fig:fnstack} that the EBP 
register lies above the EIP buffer. After overwriting the EBP buffer we 
expect to overwrite the EIP buffer. By passing a buffer of 2609 we have
overwritten the lower 4 bytes of the EIP buffer. To fill the entire buffer
would require two bytes more of information for a total of 2611 bytes. We 
have found our golden value required to overwrite the buffer. 

\begin{figure}[htbp]
    \centering
    \includegraphics[width=5.5in]{images/20160501_buffer2609.png}
    \caption{Buffer = 2609 - Lower 4 Bytes of EIP Overwritten 
      in Immunity Debugger}
    \label{fig:eipoverwrite}
\end{figure}

To test our result of we will pass the $DEVI$ into the EIP buffer. Converting
$DEVI$ into hex results in a 8 byte sequence of $0x44455649$. Knowing the
x86 processor uses little endian notation we can see our EIP register
indeed contains our expected $DEVI$ string shown in 
figure~\ref{fig:controleip}.

\begin{figure}[htbp]
    \centering
    \includegraphics[width=5.5in]{images/20160501_controlEIP.png}
    \caption{Control of EIP register with Expected String}
    \label{fig:controleip}
\end{figure}

\section{Conclusion}
\label{sect:conclusion}
We have shown that the SLMail program released in 2001 is vulnerable to 
a buffer overflow attack. Any company using this software would be vulnerable
to having their network compromised by just having the SLMail POP3 service
running. We will show in the next lab how an attacker can exploit this
vulnerability to launch a script of their choosing escalating their control
over the remote server. 

\nocite{*}
\bibliographystyle{IEEEtran}
\bibliography{20160407_project.bib}

\section*{Appendix}
\label{sect:appendix}
\lstinputlisting[language=Python, 
caption=SLMail Password Fuzzing Script,
label=lst:passfuzz]{slmail_buffer_overflow.py}

\end{document}